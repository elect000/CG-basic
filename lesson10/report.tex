% Intended LaTeX compiler: pdflatex
\documentclass{scrartcl}
		\usepackage[utf8]{inputenc}
		\usepackage[dvipdfmx]{graphicx}
		\usepackage[dvipdfmx]{color}
		\usepackage[backend=biber,bibencoding=utf8]{biblatex}
		\usepackage{url}
		\usepackage{indentfirst}
		\usepackage[normalem]{ulem}
		\usepackage{longtable}
		\usepackage{minted}
		\usepackage{fancyvrb}
    \usepackage[dvipdfmx,colorlinks=false,pdfborder={0 0 0}]{hyperref}
    \usepackage{pxjahyper}
		\bibliography{reference}
\author{情報科学類二年 江畑 拓哉(201611350)}
\date{}
\title{CG基礎 課題10}
\begin{document}

\maketitle

\section{動作環境の説明}
\label{sec:org4d2378d}
\begin{itemize}
\item OS\\
Manjaro Linux 17.0.6 Gellivara\\
\item コンパイル\\
g++ (GCC) 7.2.0\\
Copyright (C) 2017 Free Software Foundation, Inc.\\
\item コーディング\\
Spacemacs 0.200.9 (Emacs25.3.1)\\
\end{itemize}

\section{布のアニメーションを実現し、プログラムコード内の各種定数などを変更し、その挙動を確認しなさい}
\label{sec:org066abee}
\begin{itemize}
\item 課題文の定数での実行\\

\item \#define POINT\_NUM 40 の場合\\
 布の動きが滑らかになった。\\

\item \#define POINT\_NUM 60 の場合\\
 布の動きがより滑らかになった。その代わりに質点が増えたことにより布がかなり重くなり、垂れ下がりが大きくなっている。\\

\item double Ks = 4; の場合\\
 中央部の布の下がり幅が小さい。\\

\item double Ks = 10; の場合\\
 中央部の布の下がり幅が大きい。\\

\item double Mass = 25; の場合\\
 全体的に布の伸び幅が小さく見える。\\

\item double Mass = 35; の場合\\
 全体的に布の伸び幅が大きく見える。\\

\item double dT = 0.5; の場合\\
 アニメーションを遅送りにした形になっている。\\

\item double dT = 1.2; の場合\\
 アニメーションを早送りにした形になっている。\\

\item double Dk = 0.05; の場合\\
 布の波打ちが大きく見える。\\

\item double Dk = 0.2; の場合\\
 布があまり波打たず素直に落ち着いていることがわかる。\\

\item Vector3d gravity(0, -0.001, 0); の場合\\
 布の動きが緩慢になり、布の垂れ下がりも小さい。\\

\item Vector3d gravity(0, -0.004, 0); の場合\\
 布の動きが早くなり、より布が下に引っ張られていることがわかる。\\
\end{itemize}


\subsection{ソースコード}
\label{sec:org2b98f13}
変更箇所はコメントで囲っている。\\
\begin{minted}[frame=lines,linenos=true,obeytabs,tabsize=4]{c++}
#include <GL/glut.h>
#include <math.h>   
#include <stdlib.h> 
#include <stdio.h> 
#include <algorithm>
#include <vector>

// 3次元ベクトル
class Vector3d {
public:
  double x, y, z;
  Vector3d() { x = y = z = 0; }
  Vector3d(double _x, double _y, double _z) { x = _x; y = _y; z = _z; }
  void set(double _x, double _y, double _z) { x = _x; y = _y; z = _z; }
  void normalize() {
    double len = length();
    x /= len; y /= len; z /= len;
  }
  double length() { return sqrt(x * x + y * y + z * z); }
  void scale(const double s) { x *= s; y *= s; z *= s; }
  Vector3d operator+(Vector3d v) { return Vector3d(x + v.x, y + v.y, z + v.z); }
  Vector3d operator-(Vector3d v) { return Vector3d(x - v.x, y - v.y, z - v.z); }
  double operator*(Vector3d v) { return x * v.x + y* v.y + z * v.z; }
  Vector3d operator%(Vector3d v) { return Vector3d(y * v.z - z * v.y, z * v.x - x * v.z, x * v.y - y * v.x); }
  Vector3d& operator=(const Vector3d& v){ x = v.x; y = v.y; z = v.z; return (*this); }
  Vector3d& operator+=(const Vector3d& v) { x += v.x; y += v.y; z += v.z; return (*this); }
  Vector3d& operator-=(const Vector3d& v) { x -= v.x; y -= v.y; z -= v.z; return (*this); }
  void print() { printf("Vector3d(%f %f %f)\n", x, y, z); }
};
Vector3d operator-( const Vector3d& v ) { return( Vector3d( -v.x, -v.y, -v.z ) ); }
Vector3d operator*( const double& k, const Vector3d& v ) { return( Vector3d( k*v.x, k*v.y, k*v.z ) );}
Vector3d operator*( const Vector3d& v, const double& k ) { return( Vector3d( v.x*k, v.y*k, v.z*k ) );}
Vector3d operator/( const Vector3d& v, const double& k ) { return( Vector3d( v.x/k, v.y/k, v.z/k ) );}

// 質点
class Point {
public:
  Vector3d f; // 質点に働く力のベクトル
  Vector3d v; // 速度ベクトル
  Vector3d p; // 位置
  bool bFixed; // true: 固定されている false:固定されていない
};

// バネ
class Spring {
public:
  Point *p0; // 質点
  Point *p1; // 質点
  double restLength; // 自然長

  Spring(Point *_p0, Point *_p1) {
    p0 = _p0;
    p1 = _p1;
    restLength = (p0->p - p1->p).length();
  }
};

// ------------------------------------------------
#define POINT_NUM 20
// #define POINT_NUM 40
// #define POINT_NUM 60
// ------------------------------------------------

// 布の定義
class Cloth {
public:
  Point points[POINT_NUM][POINT_NUM];
  std::vector<Spring*> springs;

  Cloth() {
    // 質点の定義
    for(int y = 0; y < POINT_NUM; y++) {
      for(int x = 0; x < POINT_NUM; x++) {
        points[x][y].bFixed = false;
        // ---------------------------------------------------------------
        points[x][y].p.set((x - POINT_NUM / 2) / (POINT_NUM / 20.0),
                           (POINT_NUM / 2) / (POINT_NUM / 20.0),
                           - y / (POINT_NUM / 20.0));
        // --------------------------------------------------------------
      }
    }

    // バネの設定
    for(int y = 0; y < POINT_NUM; y++) {
      for(int x = 0; x < POINT_NUM; x++) {
        // 横方向のバネ
        if(x < POINT_NUM - 1) {				
          springs.push_back(new Spring(&points[x][y], &points[x+1][y]));
        }

        // 縦方向のバネ
        if(y < POINT_NUM -1) {
          springs.push_back(new Spring(&points[x][y], &points[x][y+1]));
        }

        // 右下方向のバネ
        if(x < POINT_NUM - 1 && y < POINT_NUM - 1) {				
          springs.push_back(new Spring(&points[x][y], &points[x+1][y+1]));
        }

        // 左下方向のバネ
        if(x > 0 && y < POINT_NUM - 1) {				
          springs.push_back(new Spring(&points[x][y], &points[x-1][y+1]));
        }
      }
    }

    // 固定点の指定
    points[0][0].bFixed = true;
    points[POINT_NUM-1][0].bFixed = true;
  }
};


Cloth *cloth;
double rotateAngleH_deg; // 画面水平方向の回転角度
double rotateAngleV_deg; // 縦方向の回転角度
int preMousePositionX;   // マウスカーソルの位置を記憶しておく変数
int preMousePositionY;   // マウスカーソルの位置を記憶しておく変数
bool bRunning; // アニメーションの実行/停止を切り替えるフラグ

// -------------------------------------------------------------------------
double Ks = 8;   // バネ定数
double Mass = 30; // 質点の質量
double dT = 1; // 時間刻み幅
double Dk = 0.1; // 速度に比例して、逆向きにはたらく抵抗係数
Vector3d gravity(0, -0.002, 0); // 重力(y軸方向の負の向きに働く)
// double Ks = 4;
// double Ks = 10;
// double Mass = 25;
// double Mass = 35;
// double dT = 0.5;
// double dT = 1.2;
// double Dk = 0.05;
// double Dk = 0.2;
// Vector3d gravity(0, -0.001, 0);
// Vector3d gravity(0, -0.004, 0);
// -------------------------------------------------------------------------

void drawCloth(void) {

  // バネを描画
  glColor3f(0.0f, 0.0f, 0.0f);
  glBegin(GL_LINES);
  for(int i = 0; i < cloth->springs.size(); i++) {
    glVertex3d(cloth->springs[i]->p0->p.x, cloth->springs[i]->p0->p.y, cloth->springs[i]->p0->p.z);
    glVertex3d(cloth->springs[i]->p1->p.x, cloth->springs[i]->p1->p.y, cloth->springs[i]->p1->p.z);
  }
  glEnd();

  // 質点を描画
  glColor3f(1.0f, 0.0f, 0.0f);
  glPointSize(4.0f);
  glBegin(GL_POINTS);
  for(int y = 0; y < POINT_NUM; y++) {
    for(int x = 0; x < POINT_NUM; x++) {
      glVertex3d(cloth->points[x][y].p.x, cloth->points[x][y].p.y, cloth->points[x][y].p.z);
    }
  }
  glEnd();
}

void display(void) {
  glClear(GL_COLOR_BUFFER_BIT | GL_DEPTH_BUFFER_BIT);
  glDisable(GL_LIGHTING);
  glLoadIdentity();
  glTranslated(0, 0.0, -50);
  glRotated(rotateAngleV_deg, 1.0, 0.0, 0.0);
  glRotated(rotateAngleH_deg, 0.0, 1.0, 0.0);
  drawCloth();

  glFlush();
}

void resize(int w, int h) {
  glViewport(0, 0, w, h);
  glMatrixMode(GL_PROJECTION);
  glLoadIdentity();
  gluPerspective(30.0, (double)w / (double)h, 1.0, 100.0);
  glMatrixMode(GL_MODELVIEW);
}

void keyboard(unsigned char key, int x, int y) {

  if (key == '\033' || key == 'q') { exit(0);} // ESC または q で終了
  if (key == 'a') { bRunning = !bRunning; }    // a でアニメーションのオンオフ
}

void mouse(int button, int state, int x, int y) {
  switch (button) {
  case GLUT_LEFT_BUTTON:
    preMousePositionX = x;
    preMousePositionY = y;
    break;
  case GLUT_MIDDLE_BUTTON:
    break;
  case GLUT_RIGHT_BUTTON:
    preMousePositionX = x;
    preMousePositionY = y;
    break;
  default:
    break;
  }
}
// -------------------------------------------------------------------
// 布の形状の更新
// 布の形状の更新
void updateCloth(void) {
  // ★次の手順で質点の位置を決定する
  // 1. 質点に働く力を求める
  // 2. 質点の加速度を求める
  // 3. 質点の速度を更新する
  // 4. 質点の位置を更新する
  //1-1. 質点に働く力をリセット
  // 全ての質点について順番に処理する
  for(int y = 0; y < POINT_NUM; y++) {
    for(int x = 0; x < POINT_NUM; x++) {
      cloth->points[x][y].f.set(0, 0, 0);
    }
  }

  //1-2. バネの両端の質点に力を働かせる
  // (考え方)質点を1つ1つ調べるのではなく
  // バネを1つ1つを見て、その両端の質点に力を加算していくアプローチをとる
  for(int i = 0; i < cloth->springs.size(); i++) {// 全てのバネについて順番に処理する
    Spring *spring = cloth->springs[i]; 

    // (a) バネの自然長と現在の長さの差分を求める
    // バネの自然長は spring->restLength に格納されている
    // 現在の長さは、両端の質点間の距離を計算して求める
    double prelen = spring->restLength;
    double newlen = (spring->p0->p - spring->p1->p).length();
    double lendiff = newlen - prelen;

    // (b) バネが質点に加える力の大きさを求める
    // (自然長 - 現在の長さ)にバネ定数 Ks を掛けた値が求める大きさ
    double powerval = Ks * lendiff;
    // (c) バネが質点に加える力の向き(単位ベクトル)を求める
    // バネには、両端に質点がついているので、一方から他方に向かう方向が力の向き
    Vector3d powervec = spring->p0->p - spring->p1->p;
    powervec.normalize();
    // (d) 両端の質点に対して、力ベクトル( 大きさは(b)で求めた。向きは(c)で求めた)を加算する
    // spring->p0->f バネの一方の質点に加わる力を表すので、これに力ベクトルを加える(向きに注意)
    // spring->p1->f バネのもう一方の質点に加わる力を表すので、これに力ベクトルを加える(向きに注意)
    spring->p0->f -= powerval * powervec;
    spring->p1->f += powerval * powervec;
  }

  //1-3. 重力、空気抵抗による力を加算する
  // 全ての質点について順番に処理する
  for(int y = 0; y < POINT_NUM; y++) {
    for(int x = 0; x < POINT_NUM; x++) {
      // cloth->points[x][y].f に、重力による力を加算する
      cloth->points[x][y].f += Mass * gravity;
      // cloth->points[x][y].f に、空気抵抗による力を加算する
      // 空気抵抗による力は速度に定数Dkをかけたもの。ただし向きは速度と逆向き
      // 速度は cloth->points[x][y].v で表される
      Vector3d newv = Dk * cloth->points[x][y].v;
      cloth->points[x][y].f -= newv;
    }
  }

  // ここまでで、質点に加わる力をすべて計算し終わった

  // 全ての質点について順番に処理する
  for(int y = 0; y < POINT_NUM; y++) {
    for(int x = 0; x < POINT_NUM; x++) {
      // 頂点が固定されている場合は何もしない
      if(cloth->points[x][y].bFixed) continue;			
      // 2. 質点の加速度(ベクトル)を計算 (力ベクトル cloth->points[x][y].f を質量で割った値)
      Vector3d acceralation = cloth->points[x][y].f / Mass;
      // 3. 質点の速度 (cloth->points[x][y].v) を加速度に基づいて更新する
      cloth->points[x][y].v += acceralation * dT;
      // 4. 質点の位置 (cloth->points[x][y].p) を速度に基づいて更新する
      cloth->points[x][y].p += cloth->points[x][y].v * dT;
      // オプション. 球体の内部に入るようなら、強制的に外に移動させる
    }
  }
}
// -------------------------------------------------------------------
void motion(int x, int y) {
  int diffX = x - preMousePositionX;
  int diffY = y - preMousePositionY;

  rotateAngleH_deg += diffX * 0.1;
  rotateAngleV_deg += diffY * 0.1;

  preMousePositionX = x;
  preMousePositionY = y;
  glutPostRedisplay();
}

// 一定時間ごとに実行される
void timer(int value) {
  if(bRunning) {
    updateCloth();
    glutPostRedisplay();
  }

  glutTimerFunc(10 , timer , 0);
}

void init(void) {
  glClearColor(1.0, 1.0, 1.0, 0.0);
  glEnable(GL_DEPTH_TEST);
  glEnable(GL_CULL_FACE);
  glEnable(GL_LIGHTING);
  glEnable(GL_LIGHT0);
}

int main(int argc, char *argv[]) {

  bRunning = true;
  cloth = new Cloth();

  glutInit(&argc, argv);
  glutInitWindowSize(600,600);
  glutInitDisplayMode(GLUT_RGBA | GLUT_DEPTH);
  glutCreateWindow(argv[0]);
  glutDisplayFunc(display);
  glutReshapeFunc(resize);
  glutKeyboardFunc(keyboard);
  glutMouseFunc(mouse);
  glutMotionFunc(motion);
  glutTimerFunc(10 , timer , 0);

  init();
  glutMainLoop();
  return 0;
}
\end{minted}
\section{オプション課題}
\label{sec:org8f7e648}
 質点が増えるごとに滑らかな動きになっていることがわかる。\\
\begin{itemize}
\item 質点 20\\
\item 質点 40\\
\item 質点 60\\
\end{itemize}
\subsection{ソースコード}
\label{sec:orgc692100}
変更箇所はコメントで囲っている。\\
\begin{minted}[frame=lines,linenos=true,obeytabs,tabsize=4]{c++}
#include <GL/glut.h>
#include <math.h>   
#include <stdlib.h> 
#include <stdio.h> 
#include <algorithm>
#include <vector>

// 3次元ベクトル
class Vector3d {
public:
  double x, y, z;
  Vector3d() { x = y = z = 0; }
  Vector3d(double _x, double _y, double _z) { x = _x; y = _y; z = _z; }
  void set(double _x, double _y, double _z) { x = _x; y = _y; z = _z; }
  void normalize() {
    double len = length();
    x /= len; y /= len; z /= len;
  }
  double length() { return sqrt(x * x + y * y + z * z); }
  void scale(const double s) { x *= s; y *= s; z *= s; }
  Vector3d operator+(Vector3d v) { return Vector3d(x + v.x, y + v.y, z + v.z); }
  Vector3d operator-(Vector3d v) { return Vector3d(x - v.x, y - v.y, z - v.z); }
  double operator*(Vector3d v) { return x * v.x + y* v.y + z * v.z; }
  Vector3d operator%(Vector3d v) { return Vector3d(y * v.z - z * v.y, z * v.x - x * v.z, x * v.y - y * v.x); }
  Vector3d& operator=(const Vector3d& v){ x = v.x; y = v.y; z = v.z; return (*this); }
  Vector3d& operator+=(const Vector3d& v) { x += v.x; y += v.y; z += v.z; return (*this); }
  Vector3d& operator-=(const Vector3d& v) { x -= v.x; y -= v.y; z -= v.z; return (*this); }
  void print() { printf("Vector3d(%f %f %f)\n", x, y, z); }
};
Vector3d operator-( const Vector3d& v ) { return( Vector3d( -v.x, -v.y, -v.z ) ); }
Vector3d operator*( const double& k, const Vector3d& v ) { return( Vector3d( k*v.x, k*v.y, k*v.z ) );}
Vector3d operator*( const Vector3d& v, const double& k ) { return( Vector3d( v.x*k, v.y*k, v.z*k ) );}
Vector3d operator/( const Vector3d& v, const double& k ) { return( Vector3d( v.x/k, v.y/k, v.z/k ) );}

// ------------------------------------------------------
// 球体
class Sphere {
public:
  Vector3d center; // 中心座標
  double radius;   // 半径

  Sphere(double x, double y, double z, double r) {
    center.x = x;
    center.y = y;
    center.z = z;
    radius = r;
  }
  void display() { 
    glPushMatrix(); // 現在のモデル変換行列を退避しておく
    glColor3f(0.1,0.1,0.1);
    // 座標の平行移動とスケール変換を施して球体を描画する
    glTranslated(center.x, center.y, center.z);
    glScaled(1, 1, 1);
    glRotated(90, 1.0, 0.0, 0.0);
    glutWireSphere(radius, 32, 32);

    glPopMatrix();  // 退避していたモデル変換行列を戻す
  }
};
double radius = 4.0;
Sphere sphere = Sphere(0, 0, 0, radius);
// ------------------------------------------------------
// 質点
class Point {
public:
  Vector3d f; // 質点に働く力のベクトル
  Vector3d v; // 速度ベクトル
  Vector3d p; // 位置
  bool bFixed; // true: 固定されている false:固定されていない
};

// バネ
class Spring {
public:
  Point *p0; // 質点
  Point *p1; // 質点
  double restLength; // 自然長

  Spring(Point *_p0, Point *_p1) {
    p0 = _p0;
    p1 = _p1;
    restLength = (p0->p - p1->p).length();
  }
};

// ----------------------------------------------------------------
#define POINT_NUM 20
// #define POINT_NUM 40
// #define POINT_NUM 60
// ----------------------------------------------------------------

// 布の定義
class Cloth {
public:
  Point points[POINT_NUM][POINT_NUM];
  std::vector<Spring*> springs;

  Cloth() {
    // 質点の定義
    for(int y = 0; y < POINT_NUM; y++) {
      for(int x = 0; x < POINT_NUM; x++) {
        points[x][y].bFixed = false;
        // ---------------------------------------------------------------
        points[x][y].p.set((x - POINT_NUM / 2) / (POINT_NUM / 20.0),
                           (POINT_NUM / 2) / (POINT_NUM / 20.0),
                           - y / (POINT_NUM / 20.0));
        // --------------------------------------------------------------
      }
    }
    // バネの設定
    for(int y = 0; y < POINT_NUM; y++) {
      for(int x = 0; x < POINT_NUM; x++) {
        // 横方向のバネ
        if(x < POINT_NUM - 1) {				
          springs.push_back(new Spring(&points[x][y], &points[x+1][y]));
        }
        // 縦方向のバネ
        if(y < POINT_NUM -1) {
          springs.push_back(new Spring(&points[x][y], &points[x][y+1]));
        }
        // 右下方向のバネ
        if(x < POINT_NUM - 1 && y < POINT_NUM - 1) {				
          springs.push_back(new Spring(&points[x][y], &points[x+1][y+1]));
        }
        // 左下方向のバネ
        if(x > 0 && y < POINT_NUM - 1) {				
          springs.push_back(new Spring(&points[x][y], &points[x-1][y+1]));
        }
      }
    }

    // 固定点の指定
    points[0][0].bFixed = true;
    points[POINT_NUM-1][0].bFixed = true;
  }
};


Cloth *cloth;
double rotateAngleH_deg; // 画面水平方向の回転角度
double rotateAngleV_deg; // 縦方向の回転角度
int preMousePositionX;   // マウスカーソルの位置を記憶しておく変数
int preMousePositionY;   // マウスカーソルの位置を記憶しておく変数
bool bRunning; // アニメーションの実行/停止を切り替えるフラグ

double Ks = 8;   // バネ定数
double Mass = 30; // 質点の質量
double dT = 1; // 時間刻み幅
double Dk = 0.1; // 速度に比例して、逆向きにはたらく抵抗係数
Vector3d gravity(0, -0.002, 0); // 重力(y軸方向の負の向きに働く)

void drawCloth(void) {

  // バネを描画
  glColor3f(0.0f, 0.0f, 0.0f);
  glBegin(GL_LINES);
  for(int i = 0; i < cloth->springs.size(); i++) {
    glVertex3d(cloth->springs[i]->p0->p.x, cloth->springs[i]->p0->p.y, cloth->springs[i]->p0->p.z);
    glVertex3d(cloth->springs[i]->p1->p.x, cloth->springs[i]->p1->p.y, cloth->springs[i]->p1->p.z);
  }
  glEnd();

  // 質点を描画
  glColor3f(1.0f, 0.0f, 0.0f);
  glPointSize(4.0f);
  glBegin(GL_POINTS);
  for(int y = 0; y < POINT_NUM; y++) {
    for(int x = 0; x < POINT_NUM; x++) {
      glVertex3d(cloth->points[x][y].p.x, cloth->points[x][y].p.y, cloth->points[x][y].p.z);
    }
  }
  glEnd();
}

void display(void) {
  glClear(GL_COLOR_BUFFER_BIT | GL_DEPTH_BUFFER_BIT);
  glDisable(GL_LIGHTING);
  glLoadIdentity();
  glTranslated(0, 0.0, -50);
  glRotated(rotateAngleV_deg, 1.0, 0.0, 0.0);
  glRotated(rotateAngleH_deg, 0.0, 1.0, 0.0);
  sphere.display();
  drawCloth();

  glFlush();
}

void resize(int w, int h) {
  glViewport(0, 0, w, h);
  glMatrixMode(GL_PROJECTION);
  glLoadIdentity();
  gluPerspective(30.0, (double)w / (double)h, 1.0, 100.0);
  glMatrixMode(GL_MODELVIEW);
}

void keyboard(unsigned char key, int x, int y) {

  if (key == '\033' || key == 'q') { exit(0);} // ESC または q で終了
  if (key == 'a') { bRunning = !bRunning; }    // a でアニメーションのオンオフ
}

void mouse(int button, int state, int x, int y) {
  switch (button) {
  case GLUT_LEFT_BUTTON:
    preMousePositionX = x;
    preMousePositionY = y;
    break;
  case GLUT_MIDDLE_BUTTON:
    break;
  case GLUT_RIGHT_BUTTON:
    preMousePositionX = x;
    preMousePositionY = y;
    break;
  default:
    break;
  }
}
// -------------------------------------------------------------------
// 布の形状の更新
// 布の形状の更新
void updateCloth(void) {
  // ★次の手順で質点の位置を決定する
  // 1. 質点に働く力を求める
  // 2. 質点の加速度を求める
  // 3. 質点の速度を更新する
  // 4. 質点の位置を更新する

  //1-1. 質点に働く力をリセット
  // 全ての質点について順番に処理する
  for(int y = 0; y < POINT_NUM; y++) {
    for(int x = 0; x < POINT_NUM; x++) {
      cloth->points[x][y].f.set(0, 0, 0);
    }
  }

  //1-2. バネの両端の質点に力を働かせる
  // (考え方)質点を1つ1つ調べるのではなく
  // バネを1つ1つを見て、その両端の質点に力を加算していくアプローチをとる
  for(int i = 0; i < cloth->springs.size(); i++) {// 全てのバネについて順番に処理する
    Spring *spring = cloth->springs[i]; 

    // (a) バネの自然長と現在の長さの差分を求める
    // バネの自然長は spring->restLength に格納されている
    // 現在の長さは、両端の質点間の距離を計算して求める
    double prelen = spring->restLength;
    double newlen = (spring->p0->p - spring->p1->p).length();
    double lendiff = newlen - prelen;

    // (b) バネが質点に加える力の大きさを求める
    // (自然長 - 現在の長さ)にバネ定数 Ks を掛けた値が求める大きさ
    double powerval = Ks * lendiff;
    // (c) バネが質点に加える力の向き(単位ベクトル)を求める
    // バネには、両端に質点がついているので、一方から他方に向かう方向が力の向き
    Vector3d powervec = spring->p0->p - spring->p1->p;
    powervec.normalize();
    // (d) 両端の質点に対して、力ベクトル( 大きさは(b)で求めた。向きは(c)で求めた)を加算する
    // spring->p0->f バネの一方の質点に加わる力を表すので、これに力ベクトルを加える(向きに注意)
    // spring->p1->f バネのもう一方の質点に加わる力を表すので、これに力ベクトルを加える(向きに注意)
    spring->p0->f -= powerval * powervec;
    spring->p1->f += powerval * powervec;
  }

  //1-3. 重力、空気抵抗による力を加算する
  // 全ての質点について順番に処理する
  for(int y = 0; y < POINT_NUM; y++) {
    for(int x = 0; x < POINT_NUM; x++) {
      // cloth->points[x][y].f に、重力による力を加算する
      cloth->points[x][y].f += Mass * gravity;
      // cloth->points[x][y].f に、空気抵抗による力を加算する
      // 空気抵抗による力は速度に定数Dkをかけたもの。ただし向きは速度と逆向き
      // 速度は cloth->points[x][y].v で表される
      Vector3d newv = Dk * cloth->points[x][y].v;
      cloth->points[x][y].f -= newv;
    }
  }

  // ここまでで、質点に加わる力をすべて計算し終わった

  // 全ての質点について順番に処理する
  for(int y = 0; y < POINT_NUM; y++) {
    for(int x = 0; x < POINT_NUM; x++) {
      // 頂点が固定されている場合は何もしない
      // if(cloth->points[x][y].bFixed) continue;			
      // 2. 質点の加速度(ベクトル)を計算 (力ベクトル cloth->points[x][y].f を質量で割った値)
      Vector3d acceralation = cloth->points[x][y].f / Mass;
      // 3. 質点の速度 (cloth->points[x][y].v) を加速度に基づいて更新する
      cloth->points[x][y].v += acceralation * dT;
      // 4. 質点の位置 (cloth->points[x][y].p) を速度に基づいて更新する
      cloth->points[x][y].p += cloth->points[x][y].v * dT;
      // オプション. 球体の内部に入るようなら、強制的に外に移動させる
      if (cloth->points[x][y].p.length() < radius) {
        cloth->points[x][y].p -= cloth->points[x][y].v * dT;
        cloth->points[x][y].v = Vector3d(0,0,0);
      }
    }
  }
}
// -------------------------------------------------------------------
void motion(int x, int y) {
  int diffX = x - preMousePositionX;
  int diffY = y - preMousePositionY;

  rotateAngleH_deg += diffX * 0.1;
  rotateAngleV_deg += diffY * 0.1;

  preMousePositionX = x;
  preMousePositionY = y;
  glutPostRedisplay();
}

// 一定時間ごとに実行される
void timer(int value) {
  if(bRunning) {
    updateCloth();
    glutPostRedisplay();
  }

  glutTimerFunc(10 , timer , 0);
}

void init(void) {
  glClearColor(1.0, 1.0, 1.0, 0.0);
  glEnable(GL_DEPTH_TEST);
  glEnable(GL_CULL_FACE);
  glEnable(GL_LIGHTING);
  glEnable(GL_LIGHT0);
}

int main(int argc, char *argv[]) {

  bRunning = true;
  cloth = new Cloth();

  glutInit(&argc, argv);
  glutInitWindowSize(600,600);
  glutInitDisplayMode(GLUT_RGBA | GLUT_DEPTH);
  glutCreateWindow(argv[0]);
  glutDisplayFunc(display);
  glutReshapeFunc(resize);
  glutKeyboardFunc(keyboard);
  glutMouseFunc(mouse);
  glutMotionFunc(motion);
  glutTimerFunc(10 , timer , 0);

  init();
  glutMainLoop();
  return 0;
}
\end{minted}
\end{document}
