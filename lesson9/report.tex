% Intended LaTeX compiler: pdflatex
\documentclass{scrartcl}
		\usepackage[utf8]{inputenc}
		\usepackage[dvipdfmx]{graphicx}
		\usepackage[backend=biber,bibencoding=utf8]{biblatex}
		\usepackage{url}
		\usepackage{indentfirst}
    \usepackage{minted}
    \usepackage[dvipdfmx]{color}
		\bibliography{reference}
\author{情報科学類二年 江畑 拓哉(201611350)}
\date{}
\title{CG基礎 課題9}
\begin{document}

\maketitle

\section{動作環境の説明}
\label{sec:org4c8a8d4}
\begin{itemize}
\item OS\\
Manjaro Linux 17.0.6 Gellivara\\
\item コンパイル\\
g++ (GCC) 7.2.0\\
Copyright (C) 2017 Free Software Foundation, Inc.\\
\item コーディング\\
Spacemacs 0.200.9 (Emacs25.3.1)\\
\end{itemize}


\section{課題1〜3}
\label{sec:org6f7a2f2}
\begin{enumerate}
\item 球と床板をレンダリングしなさい。\\
\item 床に格子模様を付加しなさい。\\
\item 床に球体の影が表示されるようにしなさい。\\
\end{enumerate}

\begin{center}
\includegraphics[width=8cm]{/mnt/c/Users/elect/Pictures/01.png}
\end{center}

\subsection{ソースコード}
\label{sec:org4fd4944}
編集部分はコメントで囲ってある。\\
\begin{minted}[frame=lines,linenos=true]{c++}
#include <GL/glut.h>
#include <math.h>
#include <stdlib.h>
#include <stdio.h>
#include <algorithm>

class Vector3d {
public:
  double x, y, z;
  Vector3d() { x = y = z = 0; }
  Vector3d(double _x, double _y, double _z) { x = _x; y = _y; z = _z; }
  void set(double _x, double _y, double _z) { x = _x; y = _y; z = _z; }
  void normalize() {
    double len = length();
    x /= len; y /= len; z /= len;
  }
  double length() { return sqrt(x * x + y * y + z * z); }
  void scale(const double s) { x *= s; y *= s; z *= s; }
  Vector3d operator+(Vector3d v) { return Vector3d(x + v.x, y + v.y, z + v.z); }
  Vector3d operator-(Vector3d v) { return Vector3d(x - v.x, y - v.y, z - v.z); }
  double operator*(Vector3d v) { return x * v.x + y* v.y + z * v.z; }
  Vector3d operator%(Vector3d v) { return Vector3d(y * v.z - z * v.y, z * v.x - x * v.z, x * v.y - y * v.x); }
  Vector3d& operator=(const Vector3d& v){ x = v.x; y = v.y; z = v.z; return (*this); }
  Vector3d& operator+=(const Vector3d& v) { x += v.x; y += v.y; z += v.z; return (*this); }
  Vector3d& operator-=(const Vector3d& v) { x -= v.x; y -= v.y; z -= v.z; return (*this); }
  void print() { printf("Vector3d(%f %f %f)\n", x, y, z); }
};
Vector3d operator-( const Vector3d& v ) { return( Vector3d( -v.x, -v.y, -v.z ) ); }
Vector3d operator*( const double& k, const Vector3d& v ) { return( Vector3d( k*v.x, k*v.y, k*v.z ) );}
Vector3d operator*( const Vector3d& v, const double& k ) { return( Vector3d( v.x*k, v.y*k, v.z*k ) );}
Vector3d operator/( const Vector3d& v, const double& k ) { return( Vector3d( v.x/k, v.y/k, v.z/k ) );}


// 球体
class Sphere {
public:
  Vector3d center; // 中心座標
  double radius;   // 半径
  double cR, cG, cB;  // Red, Green, Blue 値 0.0〜1.0

  Sphere(double x, double y, double z, double r,
    double cr, double cg, double cb) {
      center.x = x;
      center.y = y;
      center.z = z;
      radius = r;
      cR = cr;
      cG = cg;
      cB = cb;
  }

  // 点pを通り、v方向のRayとの交わりを判定する。
  // 交点が p+tv として表せる場合の t の値を返す。交わらない場合は-1を返す
  double getIntersec(Vector3d &p, Vector3d &v) {
    // A*t^2 + B*t + C = 0 の形で表す
    double A = v.x * v.x + v.y * v.y + v.z * v.z;
    double B = 2.0 * (p.x * v.x - v.x * center.x +
      p.y * v.y - v.y * center.y +
      p.z * v.z- v.z * center.z);
    double C = p.x * p.x - 2 * p.x * center.x + center.x * center.x +
      p.y * p.y - 2 * p.y * center.y + center.y * center.y +
      p.z * p.z - 2 * p.z * center.z + center.z * center.z -
      radius * radius;
    double D = B * B - 4 * A * C; // 判別式

    if (D >= 0) { // 交わる
      double t1 = (-B - sqrt(D)) / (2.0 * A);
      double t2 = (-B + sqrt(D)) / (2.0 * A);
      return t1 < t2 ? t1 : t2; // 小さいほうのtの値を返す
    } else { // 交わらない
      return -1.0;
    }
  }
};

// 板。xz平面に平行な面とする
class Board {
public:
  double y; // y座標値

  Board(double _y) {
    y = _y;
  }

  // 点pを通り、v方向のRayとの交わりを判定する。
  // 交点が p+tv として表せる場合の t の値を返す。交わらない場合は負の値を返す
  double getIntersec(Vector3d &p, Vector3d &v) {
    if(fabs(v.y) < 1.0e-10) return -1; // 水平なRayは交わらない

    double t = -1;
    // ★ここで t の値を計算する
    // ★ただしz座標が-3000より小さいなら交わらないものとする
    // -----------------------------------------------------------------------------
    t = (y + p.y) / v.y;
    if ((t < 0) || ((p + t * v).z < -3000)) {
      return -1;
    }
    // -----------------------------------------------------------------------------
    return t;
  }

  // x と z の値から床の色を返す(格子模様になるように)
  Vector3d getColorVec(double x, double z) {
    // ★ x, z の値によって(1.0, 1.0, 0.7)または(0.6, 0.6, 0.6)のどちらかの色を返すようにする
    int flag = 1;
    if (x > 0) {
      if (((int)x) % 200 > 100) flag *= -1;
    } else {
      if (((int) (-1 * x)) % 200 < 100) flag *= -1;
    }
    if (z > 0) {
      if (((int)z) % 200 > 100) flag *= -1;
    } else {
      if (((int) (-1 * z)) % 200 < 100) flag *= -1;
    }
    if (flag > 0) {
      return Vector3d(0.6, 0.6, 0.6);
    } else {
      return Vector3d(1.0, 1.0, 0.7);
    }
  }
};

int halfWidth;    // 描画領域の横幅/2
int halfHeight;   // 描画領域の縦幅/2
double screen_z = -1000;  // 投影面のz座標
double Kd = 0.8;  // 拡散反射定数
double Ks = 0.8;  // 鏡面反射定数
double Iin = 0.5; // 入射光の強さ
double Ia  = 0.5; // 環境光
Vector3d viewPosition(0, 0, 0); // 原点=視点
Vector3d lightDirection(-2, -4, -2); // 入射光の進行方向


// レンダリングする球体
Sphere sphere(0.0, 0.0, -1500, // 中心座標
        150.0,           // 半径
        0.1, 0.7, 0.7);  // RGB値

// 球体の置かれている床
Board board(-150); // y座標値を -150 にする。(球と接するようにする)

// 2つのベクトルの成す角のcos値を計算する
double getCos(Vector3d &v0, Vector3d &v1) {
  return (v0.x * v1.x + v0.y * v1.y + v0.z * v1.z ) / (v0.length() * v1.length());
}

// x, y で指定されたスクリーン座標での色(RGB)を colorVec のxyzの値に格納する
void getPixelColor(double x, double y, Vector3d &colorVec) {
  // 原点からスクリーン上のピクセルへ飛ばすレイの方向
  Vector3d ray(x - viewPosition.x, y - viewPosition.y, screen_z - viewPosition.z);
  ray.normalize(); // レイの長さの正規化
  // レイを飛ばして球と交差するか求める
  double t_sphere = sphere.getIntersec(viewPosition, ray);
  if(t_sphere > 0) { // 球との交点がある
    // ★前回の課題を参考に、球体の表面の色を計算で求め、colorVecに設定する
    double r = 0;
    double g = 0;
    double b = 0;
    // ---------------------------------------------------------------------------------
    double Id, Is, cos_Id, cos_Is, a, I;
    Is = Id = I = 0;
    Vector3d P = viewPosition + t_sphere * ray;
    Vector3d N = P - sphere.center;
    N.normalize();
    cos_Id = N * (-1 * lightDirection);
    if (cos_Id > 0) {
      Id = Iin * Kd * cos_Id;
    }
    int n = 20;
    a = -1 * (lightDirection * N);
    Vector3d R = lightDirection + 2 * a * N;
    Vector3d V = P - viewPosition;
    R.normalize();
    V.normalize();
    cos_Is = -1 * R * V;
    if (cos_Is > 0) {
      Is = Iin * Ks * pow(cos_Is, n);
    }
    I = Id + Is + Ia;
    r = std::min(I * sphere.cR, 1.0);
    g = std::min(I * sphere.cG, 1.0);
    b = std::min(I * sphere.cB, 1.0);
    // ---------------------------------------------------------------------------------
    colorVec.set(r, g, b);
    return;
  }

  // レイを飛ばして床と交差するか求める
  double t_board = board.getIntersec(viewPosition, ray);

  if(t_board > 0) { // 床との交点がある
    // ★床の表面の色を設定する
    // ★球の影になる場合は、RGBの値をそれぞれ0.5倍する
    double r = 0;
    double g = 0;
    double b = 0;
    // ----------------------------------------------------------------------------------
    double x, z;
    Vector3d board_pos = viewPosition + t_board * ray;
    Vector3d colors;
    x = board_pos.x;
    z = board_pos.z;
    colors = board.getColorVec(x, z);
    r = colors.x;
    g = colors.y;
    b = colors.z;
    if (sphere.getIntersec(board_pos, lightDirection) != -1.0) {
      r *= 0.5;
      g *= 0.5;
      b *= 0.5;
    } 
    // ----------------------------------------------------------------------------------
    colorVec.set(r, g, b);
    return;
  }

  // 何とも交差しない
  colorVec.set(0, 0, 0); // 背景色(黒)を設定する
}

// 描画を行う
void display(void) {

  glClear(GL_COLOR_BUFFER_BIT); // 描画内容のクリア

  // ピクセル単位で描画色を決定するループ処理
  for(int y = (-halfHeight); y <= halfHeight; y++ ) {
    for(int x = (-halfWidth); x <= halfWidth; x++ ) {

      Vector3d colorVec;

      // x, y 座標の色を取得する
      getPixelColor(x, y , colorVec);

      //取得した色で、描画色を設定する
      glColor3d(colorVec.x, colorVec.y, colorVec.z);

      // (x, y) の画素を描画
      glBegin(GL_POINTS);
      glVertex2i( x, y );
      glEnd();
    }
  }
  glFlush();
}

void resizeWindow(int w, int h) {
  h = (h == 0) ? 1 : h;
  glViewport(0, 0, w, h);
  halfWidth = w/2;
  halfHeight = h/2;
  glMatrixMode(GL_PROJECTION);
  glLoadIdentity();

  // ウィンドウ内の座標系設定
  glOrtho( -halfWidth, halfWidth, -halfHeight, halfHeight, 0.0, 1.0);
  glMatrixMode(GL_MODELVIEW);
}

void keyboard(unsigned char key, int x, int y) {
  switch (key) {
    case 27: exit(0);  /* ESC code */
  }
  glutPostRedisplay();
}

int main(int argc, char** argv) {
  lightDirection.normalize();

  glutInit(&argc, argv);
  glutInitDisplayMode(GLUT_SINGLE | GLUT_RGB);
  glutInitWindowSize(400,400);
  glutInitWindowPosition(180,10);
  glutCreateWindow(argv[0]);
  glClearColor(1.0, 1.0, 1.0, 1.0);
  glShadeModel(GL_FLAT);

  glutDisplayFunc(display);
  glutReshapeFunc(resizeWindow);
  glutKeyboardFunc(keyboard);
  glutMainLoop();

  return 0;

\end{minted}

\section{課題4}
\label{sec:orgfb78d34}
\begin{enumerate}
\item 鏡面を滑らかに表現するアンチエイリアシングを行うことを考える。1ピクセルに3 * 3 に 9 分割してサンプリングを行い、その9 つの値の平均値でピクセルの色を決定しなさい。\\
\end{enumerate}

\begin{center}
\includegraphics[width=8cm]{/mnt/c/Users/elect/Pictures/02.png}
\end{center}

\subsection{ソースコード}
\label{sec:orga853daf}
\begin{minted}[frame=lines,linenos=true]{c++}
#include <GL/glut.h>
#include <math.h>
#include <stdlib.h>
#include <stdio.h>
#include <algorithm>

class Vector3d {
public:
  double x, y, z;
  Vector3d() { x = y = z = 0; }
  Vector3d(double _x, double _y, double _z) { x = _x; y = _y; z = _z; }
  void set(double _x, double _y, double _z) { x = _x; y = _y; z = _z; }
  void normalize() {
    double len = length();
    x /= len; y /= len; z /= len;
  }
  double length() { return sqrt(x * x + y * y + z * z); }
  void scale(const double s) { x *= s; y *= s; z *= s; }
  Vector3d operator+(Vector3d v) { return Vector3d(x + v.x, y + v.y, z + v.z); }
  Vector3d operator-(Vector3d v) { return Vector3d(x - v.x, y - v.y, z - v.z); }
  double operator*(Vector3d v) { return x * v.x + y* v.y + z * v.z; }
  Vector3d operator%(Vector3d v) { return Vector3d(y * v.z - z * v.y, z * v.x - x * v.z, x * v.y - y * v.x); }
  Vector3d& operator=(const Vector3d& v){ x = v.x; y = v.y; z = v.z; return (*this); }
  Vector3d& operator+=(const Vector3d& v) { x += v.x; y += v.y; z += v.z; return (*this); }
  Vector3d& operator-=(const Vector3d& v) { x -= v.x; y -= v.y; z -= v.z; return (*this); }
  void print() { printf("Vector3d(%f %f %f)\n", x, y, z); }
};
Vector3d operator-( const Vector3d& v ) { return( Vector3d( -v.x, -v.y, -v.z ) ); }
Vector3d operator*( const double& k, const Vector3d& v ) { return( Vector3d( k*v.x, k*v.y, k*v.z ) );}
Vector3d operator*( const Vector3d& v, const double& k ) { return( Vector3d( v.x*k, v.y*k, v.z*k ) );}
Vector3d operator/( const Vector3d& v, const double& k ) { return( Vector3d( v.x/k, v.y/k, v.z/k ) );}


// 球体
class Sphere {
public:
  Vector3d center; // 中心座標
  double radius;   // 半径
  double cR, cG, cB;  // Red, Green, Blue 値 0.0〜1.0

  Sphere(double x, double y, double z, double r,
    double cr, double cg, double cb) {
      center.x = x;
      center.y = y;
      center.z = z;
      radius = r;
      cR = cr;
      cG = cg;
      cB = cb;
  }

  // 点pを通り、v方向のRayとの交わりを判定する。
  // 交点が p+tv として表せる場合の t の値を返す。交わらない場合は-1を返す
  double getIntersec(Vector3d &p, Vector3d &v) {
    // A*t^2 + B*t + C = 0 の形で表す
    double A = v.x * v.x + v.y * v.y + v.z * v.z;
    double B = 2.0 * (p.x * v.x - v.x * center.x +
      p.y * v.y - v.y * center.y +
      p.z * v.z- v.z * center.z);
    double C = p.x * p.x - 2 * p.x * center.x + center.x * center.x +
      p.y * p.y - 2 * p.y * center.y + center.y * center.y +
      p.z * p.z - 2 * p.z * center.z + center.z * center.z -
      radius * radius;
    double D = B * B - 4 * A * C; // 判別式

    if (D >= 0) { // 交わる
      double t1 = (-B - sqrt(D)) / (2.0 * A);
      double t2 = (-B + sqrt(D)) / (2.0 * A);
      return t1 < t2 ? t1 : t2; // 小さいほうのtの値を返す
    } else { // 交わらない
      return -1.0;
    }
  }
};

// 板。xz平面に平行な面とする
class Board {
public:
  double y; // y座標値

  Board(double _y) {
    y = _y;
  }

  // 点pを通り、v方向のRayとの交わりを判定する。
  // 交点が p+tv として表せる場合の t の値を返す。交わらない場合は負の値を返す
  double getIntersec(Vector3d &p, Vector3d &v) {
    if(fabs(v.y) < 1.0e-10) return -1; // 水平なRayは交わらない

    double t = -1;
    // ★ここで t の値を計算する
    // ★ただしz座標が-3000より小さいなら交わらないものとする
    // -----------------------------------------------------------------------------
    t = (y + p.y) / v.y;
    if ((t < 0) || ((p + t * v).z < -3000)) {
      return -1;
    }
    // -----------------------------------------------------------------------------
    return t;
  }

  // x と z の値から床の色を返す(格子模様になるように)
  Vector3d getColorVec(double x, double z) {
    // ★ x, z の値によって(1.0, 1.0, 0.7)または(0.6, 0.6, 0.6)のどちらかの色を返すようにする
    int flag = 1;
    if (x > 0) {
      if (((int)x) % 200 > 100) flag *= -1;
    } else {
      if (((int) (-1 * x)) % 200 < 100) flag *= -1;
    }
    if (z > 0) {
      if (((int)z) % 200 > 100) flag *= -1;
    } else {
      if (((int) (-1 * z)) % 200 < 100) flag *= -1;
    }
    if (flag > 0) {
      return Vector3d(0.6, 0.6, 0.6);
    } else {
      return Vector3d(1.0, 1.0, 0.7);
    }
  }
};

int halfWidth;    // 描画領域の横幅/2
int halfHeight;   // 描画領域の縦幅/2
double screen_z = -1000;  // 投影面のz座標
double Kd = 0.8;  // 拡散反射定数
double Ks = 0.8;  // 鏡面反射定数
double Iin = 0.5; // 入射光の強さ
double Ia  = 0.5; // 環境光
Vector3d viewPosition(0, 0, 0); // 原点=視点
Vector3d lightDirection(-2, -4, -2); // 入射光の進行方向


// レンダリングする球体
Sphere sphere(0.0, 0.0, -1500, // 中心座標
        150.0,           // 半径
        0.1, 0.7, 0.7);  // RGB値

// 球体の置かれている床
Board board(-150); // y座標値を -150 にする。(球と接するようにする)

// 2つのベクトルの成す角のcos値を計算する
double getCos(Vector3d &v0, Vector3d &v1) {
  return (v0.x * v1.x + v0.y * v1.y + v0.z * v1.z ) / (v0.length() * v1.length());
}

// x, y で指定されたスクリーン座標での色(RGB)を colorVec のxyzの値に格納する
void getPixelColor(double x, double y, Vector3d &colorVec) {
  // 原点からスクリーン上のピクセルへ飛ばすレイの方向
  Vector3d ray(x - viewPosition.x, y - viewPosition.y, screen_z - viewPosition.z);

  ray.normalize(); // レイの長さの正規化

  // レイを飛ばして球と交差するか求める
  double t_sphere = sphere.getIntersec(viewPosition, ray);
  if(t_sphere > 0) { // 球との交点がある
    // ★前回の課題を参考に、球体の表面の色を計算で求め、colorVecに設定する
    double r = 0;
    double g = 0;
    double b = 0;
    // ---------------------------------------------------------------------------------
    double Id, Is, cos_Id, cos_Is, a, I;
    Is = Id = I = 0;
    Vector3d P = viewPosition + t_sphere * ray;
    Vector3d N = P - sphere.center;
    N.normalize();
    cos_Id = N * (-1 * lightDirection);
    if (cos_Id > 0) {
      Id = Iin * Kd * cos_Id;
    }
    int n = 20;
    a = -1 * (lightDirection * N);
    Vector3d R = lightDirection + 2 * a * N;
    Vector3d V = P - viewPosition;
    R.normalize();
    V.normalize();
    cos_Is = -1 * R * V;
    if (cos_Is > 0) {
      Is = Iin * Ks * pow(cos_Is, n);
    }
    I = Id + Is + Ia;
    r = std::min(I * sphere.cR, 1.0);
    g = std::min(I * sphere.cG, 1.0);
    b = std::min(I * sphere.cB, 1.0);
    // ---------------------------------------------------------------------------------
    colorVec.set(r, g, b);
    return;
  }

  // レイを飛ばして床と交差するか求める
  double t_board = board.getIntersec(viewPosition, ray);

  if(t_board > 0) { // 床との交点がある
    // ★床の表面の色を設定する
    // ★球の影になる場合は、RGBの値をそれぞれ0.5倍する
    double r = 0;
    double g = 0;
    double b = 0;
    // ----------------------------------------------------------------------------------
    double x, z;
    Vector3d board_pos = viewPosition + t_board * ray;
    Vector3d colors;
    x = board_pos.x;
    z = board_pos.z;
    colors = board.getColorVec(x, z);
    r = colors.x;
    g = colors.y;
    b = colors.z;
    if (sphere.getIntersec(board_pos, lightDirection) != -1.0) {
      r *= 0.5;
      g *= 0.5;
      b *= 0.5;
    } 
    // ----------------------------------------------------------------------------------
    colorVec.set(r, g, b);
    return;
  }

  // 何とも交差しない
  colorVec.set(0, 0, 0); // 背景色(黒)を設定する
}

// 描画を行う
void display(void) {

  glClear(GL_COLOR_BUFFER_BIT); // 描画内容のクリア

  // ピクセル単位で描画色を決定するループ処理
  for(int y = (-halfHeight); y <= halfHeight; y++ ) {
    for(int x = (-halfWidth); x <= halfWidth; x++ ) {
      // ---------------------------------------------------------------------------------
      int i, j;
      Vector3d colorVecs[9];
      Vector3d colorVec;
      for (i = 1; i < 4; ++i) {
        for (j = 1; j < 4; ++j) {
          getPixelColor(x - i / 3.0, y - j / 3.0, colorVecs[(i - 1) * 3 + (j - 1)]);
        }
      }
      for (i = 0; i < 9; ++i) {
        colorVec += colorVecs[i];
      }
      //取得した色で、描画色を設定する
      glColor3d(colorVec.x / 9.0, colorVec.y / 9.0, colorVec.z / 9.0);
      // ---------------------------------------------------------------------------------

      // (x, y) の画素を描画
      glBegin(GL_POINTS);
      glVertex2i( x, y );
      glEnd();
    }
  }
  glFlush();
}

void resizeWindow(int w, int h) {
  h = (h == 0) ? 1 : h;
  glViewport(0, 0, w, h);
  halfWidth = w/2;
  halfHeight = h/2;
  glMatrixMode(GL_PROJECTION);
  glLoadIdentity();

  // ウィンドウ内の座標系設定
  glOrtho( -halfWidth, halfWidth, -halfHeight, halfHeight, 0.0, 1.0);
  glMatrixMode(GL_MODELVIEW);
}

void keyboard(unsigned char key, int x, int y) {
  switch (key) {
    case 27: exit(0);  /* ESC code */
  }
  glutPostRedisplay();
}

int main(int argc, char** argv) {
  lightDirection.normalize();

  glutInit(&argc, argv);
  glutInitDisplayMode(GLUT_SINGLE | GLUT_RGB);
  glutInitWindowSize(400,400);
  glutInitWindowPosition(180,10);
  glutCreateWindow(argv[0]);
  glClearColor(1.0, 1.0, 1.0, 1.0);
  glShadeModel(GL_FLAT);

  glutDisplayFunc(display);
  glutReshapeFunc(resizeWindow);
  glutKeyboardFunc(keyboard);
  glutMainLoop();

  return 0;
}
\end{minted}

\section{発展課題1}
\label{sec:org58e3258}
\begin{enumerate}
\item 自由にプログラムコードを変更してサンプル以外の結果を出力する。\\
\end{enumerate}

\begin{center}
\includegraphics[width=8cm]{/mnt/c/Users/elect/Pictures/03.png}
\end{center}

\subsection{ソースコード}
\label{sec:org240b49a}
\begin{minted}[frame=lines,linenos=true]{c++}
#include <GL/glut.h>
#include <math.h>
#include <stdlib.h>
#include <stdio.h>
#include <algorithm>

class Vector3d {
public:
  double x, y, z;
  Vector3d() { x = y = z = 0; }
  Vector3d(double _x, double _y, double _z) { x = _x; y = _y; z = _z; }
  void set(double _x, double _y, double _z) { x = _x; y = _y; z = _z; }
  void normalize() {
    double len = length();
    x /= len; y /= len; z /= len;
  }
  double length() { return sqrt(x * x + y * y + z * z); }
  void scale(const double s) { x *= s; y *= s; z *= s; }
  Vector3d operator+(Vector3d v) { return Vector3d(x + v.x, y + v.y, z + v.z); }
  Vector3d operator-(Vector3d v) { return Vector3d(x - v.x, y - v.y, z - v.z); }
  double operator*(Vector3d v) { return x * v.x + y* v.y + z * v.z; }
  Vector3d operator%(Vector3d v) { return Vector3d(y * v.z - z * v.y, z * v.x - x * v.z, x * v.y - y * v.x); }
  Vector3d& operator=(const Vector3d& v){ x = v.x; y = v.y; z = v.z; return (*this); }
  Vector3d& operator+=(const Vector3d& v) { x += v.x; y += v.y; z += v.z; return (*this); }
  Vector3d& operator-=(const Vector3d& v) { x -= v.x; y -= v.y; z -= v.z; return (*this); }
  void print() { printf("Vector3d(%f %f %f)\n", x, y, z); }
};
Vector3d operator-( const Vector3d& v ) { return( Vector3d( -v.x, -v.y, -v.z ) ); }
Vector3d operator*( const double& k, const Vector3d& v ) { return( Vector3d( k*v.x, k*v.y, k*v.z ) );}
Vector3d operator*( const Vector3d& v, const double& k ) { return( Vector3d( v.x*k, v.y*k, v.z*k ) );}
Vector3d operator/( const Vector3d& v, const double& k ) { return( Vector3d( v.x/k, v.y/k, v.z/k ) );}


// 球体
class Sphere {
public:
  Vector3d center; // 中心座標
  double radius;   // 半径
  double cR, cG, cB;  // Red, Green, Blue 値 0.0〜1.0

  Sphere(double x, double y, double z, double r,
    double cr, double cg, double cb) {
      center.x = x;
      center.y = y;
      center.z = z;
      radius = r;
      cR = cr;
      cG = cg;
      cB = cb;
  }

  // 点pを通り、v方向のRayとの交わりを判定する。
  // 交点が p+tv として表せる場合の t の値を返す。交わらない場合は-1を返す
  double getIntersec(Vector3d &p, Vector3d &v) {
    // A*t^2 + B*t + C = 0 の形で表す
    double A = v.x * v.x + v.y * v.y + v.z * v.z;
    double B = 2.0 * (p.x * v.x - v.x * center.x +
      p.y * v.y - v.y * center.y +
      p.z * v.z- v.z * center.z);
    double C = p.x * p.x - 2 * p.x * center.x + center.x * center.x +
      p.y * p.y - 2 * p.y * center.y + center.y * center.y +
      p.z * p.z - 2 * p.z * center.z + center.z * center.z -
      radius * radius;
    double D = B * B - 4 * A * C; // 判別式

    if (D >= 0) { // 交わる
      double t1 = (-B - sqrt(D)) / (2.0 * A);
      double t2 = (-B + sqrt(D)) / (2.0 * A);
      return t1 < t2 ? t1 : t2; // 小さいほうのtの値を返す
    } else { // 交わらない
      return -1.0;
    }
  }
};

// 板。xz平面に平行な面とする
class Board {
public:
  double y; // y座標値

  Board(double _y) {
    y = _y;
  }

  // 点pを通り、v方向のRayとの交わりを判定する。
  // 交点が p+tv として表せる場合の t の値を返す。交わらない場合は負の値を返す
  double getIntersec(Vector3d &p, Vector3d &v) {
    if(fabs(v.y) < 1.0e-10) return -1; // 水平なRayは交わらない

    double t = -1;
    // ★ここで t の値を計算する
    // ★ただしz座標が-3000より小さいなら交わらないものとする
    // -----------------------------------------------------------------------------
    t = (y + p.y) / v.y;
    if ((t < 0) || ((p + t * v).z < -3000)) {
      return -1;
    }
    // -----------------------------------------------------------------------------
    return t;
  }

  // x と z の値から床の色を返す(格子模様になるように)
  Vector3d getColorVec(double x, double z) {
    // ★ x, z の値によって(1.0, 1.0, 0.7)または(0.6, 0.6, 0.6)のどちらかの色を返すようにする
    int flag = 1;
    if (x > 0) {
      if (((int)x) % 200 > 100) flag *= -1;
    } else {
      if (((int) (-1 * x)) % 200 < 100) flag *= -1;
    }
    if (z > 0) {
      if (((int)z) % 200 > 100) flag *= -1;
    } else {
      if (((int) (-1 * z)) % 200 < 100) flag *= -1;
    }
    if (flag > 0) {
      return Vector3d(0.6, 0.6, 0.6);
    } else {
      return Vector3d(1.0, 1.0, 0.7);
    }
  }
};

int halfWidth;    // 描画領域の横幅/2
int halfHeight;   // 描画領域の縦幅/2
double screen_z = -1000;  // 投影面のz座標
double Kd = 0.8;  // 拡散反射定数
double Ks = 0.8;  // 鏡面反射定数
double Iin = 0.5; // 入射光の強さ
double Ia  = 0.5; // 環境光
Vector3d viewPosition(0, 0, 0); // 原点=視点
Vector3d lightDirection(-2, -4, -2); // 入射光の進行方向


// レンダリングする球体
Sphere sphere(0.0, 0.0, -1500, // 中心座標
        150.0,           // 半径
        0.1, 0.7, 0.7);  // RGB値

// -------------------------------------------
Sphere spheres[2] = {
  Sphere(0.0, 200, -1500,
          50.0,
          0.7, 0.1, 0.7),
  Sphere(0.0, 275, -1500,
         25.0,
         0.7, 0.7, 0.1)};
// -------------------------------------------
// 球体の置かれている床
Board board(-150); // y座標値を -150 にする。(球と接するようにする)

// 2つのベクトルの成す角のcos値を計算する
double getCos(Vector3d &v0, Vector3d &v1) {
  return (v0.x * v1.x + v0.y * v1.y + v0.z * v1.z ) / (v0.length() * v1.length());
}

// x, y で指定されたスクリーン座標での色(RGB)を colorVec のxyzの値に格納する
void getPixelColor(double x, double y, Vector3d &colorVec) {
  // 原点からスクリーン上のピクセルへ飛ばすレイの方向
  Vector3d ray(x - viewPosition.x, y - viewPosition.y, screen_z - viewPosition.z);

  ray.normalize(); // レイの長さの正規化

  // レイを飛ばして球と交差するか求める
  double t_sphere = sphere.getIntersec(viewPosition, ray);
  if(t_sphere > 0) { // 球との交点がある
    // ★前回の課題を参考に、球体の表面の色を計算で求め、colorVecに設定する
    double r = 0;
    double g = 0;
    double b = 0;
    // ---------------------------------------------------------------------------------
    double Id, Is, cos_Id, cos_Is, a, I;
    Is = Id = I = 0;
    Vector3d P = viewPosition + t_sphere * ray;
    Vector3d N = P - sphere.center;
    N.normalize();
    cos_Id = N * (-1 * lightDirection);
    if (cos_Id > 0) {
      Id = Iin * Kd * cos_Id;
    }
    int n = 20;
    a = -1 * (lightDirection * N);
    Vector3d R = lightDirection + 2 * a * N;
    Vector3d V = P - viewPosition;
    R.normalize();
    V.normalize();
    cos_Is = -1 * R * V;
    if (cos_Is > 0) {
      Is = Iin * Ks * pow(cos_Is, n);
    }
    I = Id + Is + Ia;
    r = std::min(I * sphere.cR, 1.0);
    g = std::min(I * sphere.cG, 1.0);
    b = std::min(I * sphere.cB, 1.0);
    // ---------------------------------------------------------------------------------
    colorVec.set(r, g, b);
    return;
  }
  // -----------------------------------------------------------------------------------
  for (int i = 0; i < 2; ++i) {
    t_sphere = spheres[i].getIntersec(viewPosition, ray);
    if(t_sphere > 0) {
      double r = 0;
      double g = 0;
      double b = 0;
      double Id, Is, cos_Id, cos_Is, a, I;
      Is = Id = I = 0;
      Vector3d P = viewPosition + t_sphere * ray;
      Vector3d N = P - spheres[i].center;
      N.normalize();
      cos_Id = N * (-1 * lightDirection);
      if (cos_Id > 0) {
        Id = Iin * Kd * cos_Id;
      }
      int n = 20;
      a = -1 * (lightDirection * N);
      Vector3d R = lightDirection + 2 * a * N;
      Vector3d V = P - viewPosition;
      R.normalize();
      V.normalize();
      cos_Is = -1 * R * V;
      if (cos_Is > 0) {
        Is = Iin * Ks * pow(cos_Is, n);
      }
      I = Id + Is + Ia;
      r = std::min(I * spheres[i].cR, 1.0);
      g = std::min(I * spheres[i].cG, 1.0);
      b = std::min(I * spheres[i].cB, 1.0);
      colorVec.set(r, g, b);
      return;
    }
  }

  // -----------------------------------------------------------------------------------

  // レイを飛ばして床と交差するか求める
  double t_board = board.getIntersec(viewPosition, ray);

  if(t_board > 0) { // 床との交点がある
    // ★床の表面の色を設定する
    // ★球の影になる場合は、RGBの値をそれぞれ0.5倍する
    double r = 0;
    double g = 0;
    double b = 0;
    // ----------------------------------------------------------------------------------
    double x, z;
    Vector3d board_pos = viewPosition + t_board * ray;
    Vector3d colors;
    Vector3d d_lightDirection = -1 * lightDirection;
    d_lightDirection.normalize();
    x = board_pos.x;
    z = board_pos.z;
    colors = board.getColorVec(x, z);
    r = colors.x;
    g = colors.y;
    b = colors.z;
    for (int i = 0; i < 2; ++i) {
      if (spheres[i].getIntersec(board_pos, d_lightDirection) > 0.0) {
        r *= 0.5;
        g *= 0.5;
        b *= 0.5;
      }
    }
    if (sphere.getIntersec(board_pos, d_lightDirection) > 0.0) {
      r *= 0.5;
      g *= 0.5;
      b *= 0.5;
    }
    // ----------------------------------------------------------------------------------
    colorVec.set(r, g, b);
    return;
  }

  // 何とも交差しない
  colorVec.set(0, 0, 0); // 背景色(黒)を設定する
}

// 描画を行う
void display(void) {

  glClear(GL_COLOR_BUFFER_BIT); // 描画内容のクリア

  // ピクセル単位で描画色を決定するループ処理
  for(int y = (-halfHeight); y <= halfHeight; y++ ) {
    for(int x = (-halfWidth); x <= halfWidth; x++ ) {
      // ---------------------------------------------------------------------------------
      int i, j;
      Vector3d colorVecs[9];
      Vector3d colorVec;
      for (i = 1; i < 4; ++i) {
        for (j = 1; j < 4; ++j) {
          getPixelColor(x - i / 3.0, y - j / 3.0, colorVecs[(i - 1) * 3 + (j - 1)]);
        }
      }
      for (i = 0; i < 9; ++i) {
        colorVec += colorVecs[i];
      }
      //取得した色で、描画色を設定する
      glColor3d(colorVec.x / 9.0, colorVec.y / 9.0, colorVec.z / 9.0);
      // ---------------------------------------------------------------------------------

      // (x, y) の画素を描画
      glBegin(GL_POINTS);
      glVertex2i( x, y );
      glEnd();
    }
  }
  glFlush();
}

void resizeWindow(int w, int h) {
  h = (h == 0) ? 1 : h;
  glViewport(0, 0, w, h);
  halfWidth = w/2;
  halfHeight = h/2;
  glMatrixMode(GL_PROJECTION);
  glLoadIdentity();

  // ウィンドウ内の座標系設定
  glOrtho( -halfWidth, halfWidth, -halfHeight, halfHeight, 0.0, 1.0);
  glMatrixMode(GL_MODELVIEW);
}

void keyboard(unsigned char key, int x, int y) {
  switch (key) {
    case 27: exit(0);  /* ESC code */
  }
  glutPostRedisplay();
}

int main(int argc, char** argv) {
  lightDirection.normalize();

  glutInit(&argc, argv);
  glutInitDisplayMode(GLUT_SINGLE | GLUT_RGB);
  glutInitWindowSize(400,400);
  glutInitWindowPosition(180,10);
  glutCreateWindow(argv[0]);
  glClearColor(1.0, 1.0, 1.0, 1.0);
  glShadeModel(GL_FLAT);

  glutDisplayFunc(display);
  glutReshapeFunc(resizeWindow);
  glutKeyboardFunc(keyboard);
  glutMainLoop();

  return 0;
}
\end{minted}

\section{発展課題2}
\label{sec:org88f2767}
\begin{enumerate}
\item 飛ばしたレイを球面で反射させ、鏡面反射を実現してみる。\\
\end{enumerate}

\begin{center}
\includegraphics[width=8cm]{/mnt/c/Users/elect/Pictures/04.png}
\end{center}

\subsection{ソースコード}
\label{sec:org86d9aa2}
\begin{minted}[frame=lines,linenos=true]{c++}
#include <GL/glut.h>
#include <math.h>
#include <stdlib.h>
#include <stdio.h>
#include <algorithm>

class Vector3d {
public:
  double x, y, z;
  Vector3d() { x = y = z = 0; }
  Vector3d(double _x, double _y, double _z) { x = _x; y = _y; z = _z; }
  void set(double _x, double _y, double _z) { x = _x; y = _y; z = _z; }
  void normalize() {
    double len = length();
    x /= len; y /= len; z /= len;
  }
  double length() { return sqrt(x * x + y * y + z * z); }
  void scale(const double s) { x *= s; y *= s; z *= s; }
  Vector3d operator+(Vector3d v) { return Vector3d(x + v.x, y + v.y, z + v.z); }
  Vector3d operator-(Vector3d v) { return Vector3d(x - v.x, y - v.y, z - v.z); }
  double operator*(Vector3d v) { return x * v.x + y* v.y + z * v.z; }
  Vector3d operator%(Vector3d v) { return Vector3d(y * v.z - z * v.y, z * v.x - x * v.z, x * v.y - y * v.x); }
  Vector3d& operator=(const Vector3d& v){ x = v.x; y = v.y; z = v.z; return (*this); }
  Vector3d& operator+=(const Vector3d& v) { x += v.x; y += v.y; z += v.z; return (*this); }
  Vector3d& operator-=(const Vector3d& v) { x -= v.x; y -= v.y; z -= v.z; return (*this); }
  void print() { printf("Vector3d(%f %f %f)\n", x, y, z); }
};
Vector3d operator-( const Vector3d& v ) { return( Vector3d( -v.x, -v.y, -v.z ) ); }
Vector3d operator*( const double& k, const Vector3d& v ) { return( Vector3d( k*v.x, k*v.y, k*v.z ) );}
Vector3d operator*( const Vector3d& v, const double& k ) { return( Vector3d( v.x*k, v.y*k, v.z*k ) );}
Vector3d operator/( const Vector3d& v, const double& k ) { return( Vector3d( v.x/k, v.y/k, v.z/k ) );}


// 球体
class Sphere {
public:
  Vector3d center; // 中心座標
  double radius;   // 半径
  double cR, cG, cB;  // Red, Green, Blue 値 0.0〜1.0

  Sphere(double x, double y, double z, double r,
    double cr, double cg, double cb) {
      center.x = x;
      center.y = y;
      center.z = z;
      radius = r;
      cR = cr;
      cG = cg;
      cB = cb;
  }

  // 点pを通り、v方向のRayとの交わりを判定する。
  // 交点が p+tv として表せる場合の t の値を返す。交わらない場合は-1を返す
  double getIntersec(Vector3d &p, Vector3d &v) {
    // A*t^2 + B*t + C = 0 の形で表す
    double A = v.x * v.x + v.y * v.y + v.z * v.z;
    double B = 2.0 * (p.x * v.x - v.x * center.x +
      p.y * v.y - v.y * center.y +
      p.z * v.z- v.z * center.z);
    double C = p.x * p.x - 2 * p.x * center.x + center.x * center.x +
      p.y * p.y - 2 * p.y * center.y + center.y * center.y +
      p.z * p.z - 2 * p.z * center.z + center.z * center.z -
      radius * radius;
    double D = B * B - 4 * A * C; // 判別式

    if (D >= 0) { // 交わる
      double t1 = (-B - sqrt(D)) / (2.0 * A);
      double t2 = (-B + sqrt(D)) / (2.0 * A);
      return t1 < t2 ? t1 : t2; // 小さいほうのtの値を返す
    } else { // 交わらない
      return -1.0;
    }
  }
};

// 板。xz平面に平行な面とする
class Board {
public:
  double y; // y座標値

  Board(double _y) {
    y = _y;
  }

  // 点pを通り、v方向のRayとの交わりを判定する。
  // 交点が p+tv として表せる場合の t の値を返す。交わらない場合は負の値を返す
  double getIntersec(Vector3d &p, Vector3d &v) {
    if(fabs(v.y) < 1.0e-10) return -1; // 水平なRayは交わらない

    double t = -1;
    // ★ここで t の値を計算する
    // ★ただしz座標が-3000より小さいなら交わらないものとする
    // -----------------------------------------------------------------------------
    t = (y + p.y) / v.y;
    if ((t < 0) || ((p + t * v).z < -3000)) {
      return -1;
    }
    // -----------------------------------------------------------------------------
    return t;
  }

  // x と z の値から床の色を返す(格子模様になるように)
  Vector3d getColorVec(double x, double z) {
    // ★ x, z の値によって(1.0, 1.0, 0.7)または(0.6, 0.6, 0.6)のどちらかの色を返すようにする
    int flag = 1;
    if (x > 0) {
      if (((int)x) % 200 > 100) flag *= -1;
    } else {
      if (((int) (-1 * x)) % 200 < 100) flag *= -1;
    }
    if (z > 0) {
      if (((int)z) % 200 > 100) flag *= -1;
    } else {
      if (((int) (-1 * z)) % 200 < 100) flag *= -1;
    }
    if (flag > 0) {
      return Vector3d(0.6, 0.6, 0.6);
    } else {
      return Vector3d(1.0, 1.0, 0.7);
    }
  }
};

int halfWidth;    // 描画領域の横幅/2
int halfHeight;   // 描画領域の縦幅/2
double screen_z = -1000;  // 投影面のz座標
double Kd = 0.8;  // 拡散反射定数
double Ks = 0.8;  // 鏡面反射定数
double Iin = 0.5; // 入射光の強さ
double Ia  = 0.5; // 環境光
Vector3d viewPosition(0, 0, 0); // 原点=視点
Vector3d lightDirection(-2, -4, -2); // 入射光の進行方向


// レンダリングする球体
Sphere sphere(0.0, 0.0, -1500, // 中心座標
        150.0,           // 半径
        0.1, 0.7, 0.7);  // RGB値

// 球体の置かれている床
Board board(-150); // y座標値を -150 にする。(球と接するようにする)

// 2つのベクトルの成す角のcos値を計算する
double getCos(Vector3d &v0, Vector3d &v1) {
  return (v0.x * v1.x + v0.y * v1.y + v0.z * v1.z ) / (v0.length() * v1.length());
}

// x, y で指定されたスクリーン座標での色(RGB)を colorVec のxyzの値に格納する
void getPixelColor(double x, double y, Vector3d &colorVec) {
  // 原点からスクリーン上のピクセルへ飛ばすレイの方向
  Vector3d ray(x - viewPosition.x, y - viewPosition.y, screen_z - viewPosition.z);

  ray.normalize(); // レイの長さの正規化

  // レイを飛ばして球と交差するか求める
  double t_sphere = sphere.getIntersec(viewPosition, ray);
  if(t_sphere > 0) { // 球との交点がある
    // ★前回の課題を参考に、球体の表面の色を計算で求め、colorVecに設定する
    double r = 0;
    double g = 0;
    double b = 0;
    // ---------------------------------------------------------------------------------
    double Id, Is, cos_Id, cos_Is, a, I;
    Is = Id = I = 0;
    Vector3d P = viewPosition + t_sphere * ray;
    Vector3d N = P - sphere.center;
    N.normalize();
    cos_Id = N * (-1 * lightDirection);
    if (cos_Id > 0) {
      Id = Iin * Kd * cos_Id;
    }
    int n = 20;
    a = -1 * (lightDirection * N);
    Vector3d R = lightDirection + 2 * a * N;
    Vector3d V = P - viewPosition;
    R.normalize();
    V.normalize();
    cos_Is = -1 * R * V;
    if (cos_Is > 0) {
      Is = Iin * Ks * pow(cos_Is, n);
    }
    I = Id + Is + Ia;
    r = std::min(I * sphere.cR, 1.0);
    g = std::min(I * sphere.cG, 1.0);
    b = std::min(I * sphere.cB, 1.0);
    double a2 = -1 * P * N;
    Vector3d R2 = P + 2 * a2 * N;
    double t2 = board.getIntersec(P, R2);
    if (t2 > 0.0) {
      Vector3d K = P + t2 * R2;
      Vector3d colors = board.getColorVec(K.x, K.z);
      r += colors.x;
      g += colors.y;
      b += colors.z;
      r /= 2;
      g /= 2;
      b /= 2;
    }
    // ---------------------------------------------------------------------------------
    colorVec.set(r, g, b);
    return;
  }

  // レイを飛ばして床と交差するか求める
  double t_board = board.getIntersec(viewPosition, ray);

  if(t_board > 0) { // 床との交点がある
    // ★床の表面の色を設定する
    // ★球の影になる場合は、RGBの値をそれぞれ0.5倍する
    double r = 0;
    double g = 0;
    double b = 0;
    // ----------------------------------------------------------------------------------
    double x, z;
    Vector3d board_pos = viewPosition + t_board * ray;
    Vector3d colors;
    x = board_pos.x;
    z = board_pos.z;
    colors = board.getColorVec(x, z);
    r = colors.x;
    g = colors.y;
    b = colors.z;
    if (sphere.getIntersec(board_pos, lightDirection) != -1.0) {
      r *= 0.5;
      g *= 0.5;
      b *= 0.5;
    }
    // ----------------------------------------------------------------------------------
    colorVec.set(r, g, b);
    return;
  }

  // 何とも交差しない
  colorVec.set(0, 0, 0); // 背景色(黒)を設定する
}

// 描画を行う
void display(void) {

  glClear(GL_COLOR_BUFFER_BIT); // 描画内容のクリア

  // ピクセル単位で描画色を決定するループ処理
  for(int y = (-halfHeight); y <= halfHeight; y++ ) {
    for(int x = (-halfWidth); x <= halfWidth; x++ ) {
      // ---------------------------------------------------------------------------------
      int i, j;
      Vector3d colorVecs[9];
      Vector3d colorVec;
      for (i = 1; i < 4; ++i) {
        for (j = 1; j < 4; ++j) {
          getPixelColor(x - i / 3.0, y - j / 3.0, colorVecs[(i - 1) * 3 + (j - 1)]);
        }
      }
      for (i = 0; i < 9; ++i) {
        colorVec += colorVecs[i];
      }
      //取得した色で、描画色を設定する
      glColor3d(colorVec.x / 9.0, colorVec.y / 9.0, colorVec.z / 9.0);
      // ---------------------------------------------------------------------------------

      // (x, y) の画素を描画
      glBegin(GL_POINTS);
      glVertex2i( x, y );
      glEnd();
    }
  }
  glFlush();
}

void resizeWindow(int w, int h) {
  h = (h == 0) ? 1 : h;
  glViewport(0, 0, w, h);
  halfWidth = w/2;
  halfHeight = h/2;
  glMatrixMode(GL_PROJECTION);
  glLoadIdentity();

  // ウィンドウ内の座標系設定
  glOrtho( -halfWidth, halfWidth, -halfHeight, halfHeight, 0.0, 1.0);
  glMatrixMode(GL_MODELVIEW);
}

void keyboard(unsigned char key, int x, int y) {
  switch (key) {
    case 27: exit(0);  /* ESC code */
  }
  glutPostRedisplay();
}

int main(int argc, char** argv) {
  lightDirection.normalize();

  glutInit(&argc, argv);
  glutInitDisplayMode(GLUT_SINGLE | GLUT_RGB);
  glutInitWindowSize(400,400);
  glutInitWindowPosition(180,10);
  glutCreateWindow(argv[0]);
  glClearColor(1.0, 1.0, 1.0, 1.0);
  glShadeModel(GL_FLAT);

  glutDisplayFunc(display);
  glutReshapeFunc(resizeWindow);
  glutKeyboardFunc(keyboard);
  glutMainLoop();

  return 0;
}
\end{minted}
\end{document}
