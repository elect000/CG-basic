% Intended LaTeX compiler: pdflatex
\documentclass{scrartcl}
		\usepackage[utf8]{inputenc}
		\usepackage[dvipdfmx]{graphicx}
		\usepackage[backend=biber,bibencoding=utf8]{biblatex}
		\usepackage{url}
		\usepackage{indentfirst}
		\bibliography{reference}
\author{情報科学類二年 江畑 拓哉(201611350)}
\date{}
\title{CG基礎 課題7}
\begin{document}

\maketitle

\section{動作環境の説明}
\label{sec:org100cc06}
\begin{itemize}
\item OS\\
Manjaro Linux 17.0.6 Gellivara\\
\item コンパイル\\
g++ (GCC) 7.2.0\\
Copyright (C) 2017 Free Software Foundation, Inc.\\
\item コーディング\\
Spacemacs 0.200.9 (Emacs25.3.1)\\
\end{itemize}
\section{ソースコード}
\label{sec:orgedeff83}
 はじめに今回の課題で用いたソースコードを示す。変数flagなどを変更することでそれぞれの課題の実行が行われている。\\
\begin{minted}[frame=lines,linenos=true]{c++}
#include <GL/glut.h>
#include <math.h>
#include <stdlib.h>
#include <stdio.h>
#include <algorithm> // 小さい方の値を返す std::min 関数を使うため

class Vector3d {
public:
  double x, y, z;
  Vector3d() { x = y = z = 0; }
  Vector3d(double _x, double _y, double _z) { x = _x; y = _y; z = _z; }
  void set(double _x, double _y, double _z) { x = _x; y = _y; z = _z; }

  // 長さを1に正規化する
  void normalize() {
    double len = length();
    x /= len; y /= len; z /= len;
  }

  // 長さを返す
  double length() { return sqrt(x * x + y * y + z * z); }

  // s倍する
  void scale(const double s) { x *= s; y *= s; z *= s; }

  // 加算の定義
  Vector3d operator+(Vector3d v) { return Vector3d(x + v.x, y + v.y, z + v.z); }

  // 減算の定義
  Vector3d operator-(Vector3d v) { return Vector3d(x - v.x, y - v.y, z - v.z); }

  // 内積の定義
  double operator*(Vector3d v) { return x * v.x + y* v.y + z * v.z; }

  // 外積の定義
  Vector3d operator%(Vector3d v) { return Vector3d(y * v.z - z * v.y, z * v.x - x * v.z, x * v.y - y * v.x); }

  // 代入演算の定義
  Vector3d& operator=(const Vector3d& v){ x = v.x; y = v.y; z = v.z; return (*this); }

  // 加算代入の定義
  Vector3d& operator+=(const Vector3d& v) { x += v.x; y += v.y; z += v.z; return (*this); }

  // 減算代入の定義
  Vector3d& operator-=(const Vector3d& v) { x -= v.x; y -= v.y; z -= v.z; return (*this); }

  // 値を出力する
  void print() { printf("Vector3d(%f %f %f)\n", x, y, z); }
};
// マイナスの符号の付いたベクトルを扱えるようにするための定義 例:b=(-a); のように記述できる
Vector3d operator-( const Vector3d& v ) { return( Vector3d( -v.x, -v.y, -v.z ) ); }

// ベクトルと実数の積を扱えるようにするための定義 例: c=5*a+2*b; c=b*3; のように記述できる
Vector3d operator*( const double& k, const Vector3d& v ) { return( Vector3d( k*v.x, k*v.y, k*v.z ) );}
Vector3d operator*( const Vector3d& v, const double& k ) { return( Vector3d( v.x*k, v.y*k, v.z*k ) );}

// ベクトルを実数で割る操作を扱えるようにするための定義 例: c=a/2.3; のように記述できる
Vector3d operator/( const Vector3d& v, const double& k ) { return( Vector3d( v.x/k, v.y/k, v.z/k ) );}


// 球体
class Sphere {
public:
  Vector3d center; // 中心座標
  double radius;   // 半径
  double cR, cG, cB;  // Red, Green, Blue 値 0.0〜1.0

  Sphere(double x, double y, double z, double r,
    double cr, double cg, double cb) {
      center.x = x;
      center.y = y;
      center.z = z;
      radius = r;
      cR = cr;
      cG = cg;
      cB = cb;
  }

  // 点pを通り、v方向のRayとの交わりを判定する。
  // 交点が p+tv として表せる場合の t の値を返す。交わらない場合は-1を返す
  double getIntersec(Vector3d &p, Vector3d &v) {
    // A*t^2 + B*t + C = 0 の形で表す
    double A = v.x * v.x + v.y * v.y + v.z * v.z;
    double B = 2.0 * (p.x * v.x - v.x * center.x +
      p.y * v.y - v.y * center.y +
      p.z * v.z- v.z * center.z);
    double C = p.x * p.x - 2 * p.x * center.x + center.x * center.x +
      p.y * p.y - 2 * p.y * center.y + center.y * center.y +
      p.z * p.z - 2 * p.z * center.z + center.z * center.z -
      radius * radius;
    double D = B * B - 4 * A * C; // 判別式

    if (D >= 0) { // 交わる
      double t1 = (-B - sqrt(D)) / (2.0 * A);
      double t2 = (-B + sqrt(D)) / (2.0 * A);
      return t1 < t2 ? t1 : t2; // 小さいほうのtの値を返す
    } else { // 交わらない
      return -1.0;
    }
  }
};


int halfWidth;    // 描画領域の横幅/2
int halfHeight;   // 描画領域の縦幅/2

// 各種定数
double d = 1000;  // 視点と投影面との距離
double Kd = 0.8;  // 拡散反射定数
// double Kd = 0.6;  // 8-3-8
// double Kd = 1.0;  // 8-3-9
double Ks = 0.8;  // 鏡面反射定数
// double Ks = 0.6;  // 8-3-10
// double Ks = 1.0;  // 8-3-11
// double Ks = 4.0;  // 8-3-12
double Iin = 1.0; // 入射光の強さ
// double Iin = 0.5; // 8-3-6
// double Iin = 2.0; // 8-3-7
double Ia  = 0.2; // 環境光
// double Ia  = 0.1; // 8-3-4
// double Ia  = 0.4; // 8-3-5

Vector3d viewPosition(0, 0, 0); // 視点位置
Vector3d lightDirection(-2, -4, -2); // 入射光の進行方向
// Vector3d lightDirection(0, -4, -2); // 8-3-1
// Vector3d lightDirection(0, -4, -2); // 8-3-2
// Vector3d lightDirection(-2, 0, -2); // 8-3-3

// レンダリングする球体
Sphere sphere(0.0, 0.0, -1500, // 中心座標
        150.0,           // 半径
        0.2, 0.9, 0.9);  // RGB値


// 描画を行う
void display(void) {

  glClear(GL_COLOR_BUFFER_BIT); // 描画内容のクリア

  // ピクセル単位で描画色を決定するループ処理
  for(int y = (-halfHeight); y <= halfHeight; y++ ) {
    for(int x = (-halfWidth); x <= halfWidth; x++ ) {

      Vector3d ray(x - viewPosition.x, y - viewPosition.y, -d - viewPosition.z); // 原点からスクリーン上のピクセルへ飛ばすレイの方向
      ray.normalize(); // レイの長さの正規化

      // レイを飛ばして球との交点を求める
      double t = sphere.getIntersec(viewPosition, ray);

      if(t > 0) { // 交点がある
        double Is = 0; // 鏡面反射光
        double Id = 0; // 拡散反射光
        // ----------------------------------------------------------------------------------
        // ★ここで Is および Id の値を計算する
        int flag = 1; // 0: 拡散反射光 <=> 1: 鏡面反射光
        Vector3d P = viewPosition + t * ray;
        Vector3d N = P - sphere.center;
        N.normalize();
        if (flag >= 0) {
          double cos_Id = N * (-1 * lightDirection);
          if(cos_Id > 0) {
            Id = Iin * Kd * cos_Id;
          }
        } if(flag >= 1) {
          int n = 20;
          // int n = 2; // 8-3-13
          // int n = 40; // 8-3-14
          double a = -1 * (lightDirection * N);
          Vector3d R = lightDirection + 2 * a * N;
          Vector3d V = P - viewPosition;
          R.normalize();
          V.normalize();
          double cos_Is = -1 * R * V;
          if (cos_Is > 0) {
            Is = Iin * Ks * pow(cos_Is, n);
          }
        }
        // ---------------------------------------------------------------------------------
        double I = Id + Is + Ia;
        double r = std::min(I * sphere.cR, 1.0); // 1.0 を超えないようにする
        double g = std::min(I * sphere.cG, 1.0); // 1.0 を超えないようにする
        double b = std::min(I * sphere.cB, 1.0); // 1.0 を超えないようにする

        // 描画色の設定
        glColor3d(r, g, b);

      } else { // 交点が無い

        // 描画色を黒にする
        glColor3f(0.0f, 0.0f, 0.0f);
      }

      // (x, y) の画素を描画
      glBegin(GL_POINTS);
      glVertex2i(x, y);
      glEnd();
    }
  }
  glFlush();
}

void resize(int w, int h) {
  if (h < 1) return;
  glViewport(0, 0, w, h);
  halfWidth = w/2;
  halfHeight = h/2;
  glMatrixMode(GL_PROJECTION);
  glLoadIdentity();

  // ウィンドウ内の座標系設定
  glOrtho( -halfWidth, halfWidth, -halfHeight, halfHeight, 0.0, 1.0);
  glMatrixMode(GL_MODELVIEW);
}

void keyboard(unsigned char key, int x, int y) {
  switch (key) {
    case 27: exit(0);  /* ESC code */
  }
  glutPostRedisplay();
}

int main(int argc, char** argv) {
  lightDirection.normalize();

  glutInit(&argc, argv);
  glutInitDisplayMode(GLUT_SINGLE | GLUT_RGB);
  glutInitWindowSize(400,400);
  glutCreateWindow(argv[0]);
  glClearColor(1.0, 1.0, 1.0, 1.0);
  glShadeModel(GL_FLAT);

  glutDisplayFunc(display);
  glutReshapeFunc(resize);
  glutKeyboardFunc(keyboard);
  glutMainLoop();

  return 0;

}
\end{minted}
\section{課題1}
\label{sec:org4d88458}
153行目の変数flagを0にすることで実行可能である。\\
\begin{center}
\includegraphics[width=6cm]{/home/elect/Pictures/8-1.png}
\end{center}
\section{課題2}
\label{sec:orgbdcf604}
153行目の変数flagを1にすることで実行可能である。\\
\begin{center}
\includegraphics[width=6cm]{/home/elect/Pictures/8-2.png}
\end{center}
\section{課題3}
\label{sec:orgc44da04}
\subsection{1}
\label{sec:org1a4500f}
125行目 \texttt{Vector3d lightDirection(0, 0, -2); // 8-3-1} を有効にした場合、入射光が手前からのものになった。\\
\begin{center}
\includegraphics[width=6cm]{/home/elect/Pictures/8-3-1.png}
\end{center}
\subsection{2}
\label{sec:orgdc10422}
126行目 \texttt{Vector3d lightDirection(0, -4, -2); // 8-3-2} を有効にした場合、入射光が上からのものになった。\\
\begin{center}
\includegraphics[width=6cm]{/home/elect/Pictures/8-3-2.png}
\end{center}
\subsection{3}
\label{sec:orgc103b08}
127行目 \texttt{Vector3d lightDirection(-2, 0, -2); // 8-3-3} を有効にした場合、入射光が右からのものになった。\\
\begin{center}
\includegraphics[width=6cm]{/home/elect/Pictures/8-3-3.png}
\end{center}
\subsection{4}
\label{sec:org8f8b637}
120行目 \texttt{double Ia  = 0.1; // 8-3-4} を有効にした場合、全体的に暗くなった。\\
\begin{center}
\includegraphics[width=6cm]{/home/elect/Pictures/8-3-4.png}
\end{center}
\subsection{5}
\label{sec:org747fea3}
121行目 \texttt{double Ia  = 0.4; // 8-3-5} を有効にした場合、全体的に明るくなった。\\
\begin{center}
\includegraphics[width=6cm]{/home/elect/Pictures/8-3-5.png}
\end{center}
\subsection{6}
\label{sec:org7103a06}
117行目 \texttt{double Iin = 0.5; // 8-3-6} を有効にした場合、全体的に暗くなった。(入射光が暗くなったが、影も当然暗いため全体的に暗く感じた)\\
\begin{center}
\includegraphics[width=6cm]{/home/elect/Pictures/8-3-6.png}
\end{center}
\subsection{7}
\label{sec:org5f16c2b}
118行目 \texttt{double Iin = 2.0; // 8-3-6} を有効にした場合、入射光が強くなった。\\
\begin{center}
\includegraphics[width=6cm]{/home/elect/Pictures/8-3-7.png}
\end{center}
\subsection{8}
\label{sec:orgbdc51eb}
110行目 \texttt{double Kd = 0.6;  // 8-3-8} を有効にした場合、鏡面部分を除いて暗くなった。\\
\begin{center}
\includegraphics[width=6cm]{/home/elect/Pictures/8-3-8.png}
\end{center}
\subsection{9}
\label{sec:orgb76d250}
111行目 \texttt{double Kd = 1.0;  // 8-3-9} を有効にした場合、鏡面部分を除いて明るくなった。\\
\begin{center}
\includegraphics[width=6cm]{/home/elect/Pictures/8-3-9.png}
\end{center}
\subsection{10}
\label{sec:orge9424cc}
113行目 \texttt{double Ks = 0.6;  // 8-3-10} を有効にした場合、鏡面部分がほんのり暗くなった。\\
\begin{center}
\includegraphics[width=6cm]{/home/elect/Pictures/8-3-10.png}
\end{center}
\subsection{11}
\label{sec:org49b7fb8}
114行目 \texttt{double Ks = 1.0;  // 8-3-11} を有効にした場合、鏡面部分がほんのり明るくなった。\\
\begin{center}
\includegraphics[width=6cm]{/home/elect/Pictures/8-3-11.png}
\end{center}
\subsection{12}
\label{sec:org8063546}
上の変化が少なかったため、追加した。115行目 \texttt{double Ks = 4.0;  // 8-3-12} を有効にした場合、鏡面部分がはっきり明るくなった。\\
\begin{center}
\includegraphics[width=6cm]{/home/elect/Pictures/8-3-12.png}
\end{center}
\subsection{13}
\label{sec:orgce57054}
166行目 \texttt{int n = 2; // 8-3-13} を有効にした場合、鏡面部分が潰れてぼやけて見えた。\\
\begin{center}
\includegraphics[width=6cm]{/home/elect/Pictures/8-3-13.png}
\end{center}
\subsection{14}
\label{sec:org1e617ea}
167行目 \texttt{int n = 40; // 8-3-14} を有効にした場合、鏡面部分が鋭くはっきりと見えた。\\
\begin{center}
\includegraphics[width=6cm]{/home/elect/Pictures/8-3-14.png}
\end{center}
\end{document}
